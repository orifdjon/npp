\chapter{Программная реализация КАМ-машины}

\begin{annotation}
	В данном разделе представляется реализация, отлаживание, тестирование настольного приложения КАМ-машины.
	Приводятся производительность каждого компонента. Приведены фрагменты программного кода. Настольное приложение написано на платформе .NET. Обосновывается выбор инструментарий для реализации настольного приложения. Веб-сервис реализуется при помощи ASP.NET на сервере и Vue js на клиенте.
\end{annotation}

\section{Обосновние выбора инструментальных средств для разработки}

\begin{annotation}
	В данном подразделе описываются причины выбора фреймворка .NET и языков C Sharp и F Sharp, а также таких библиотек и вспомогательного ПО как Mercurial и NUnit.
\end{annotation}

Как известно, необходимо было реализовать логику работы абстрактной машины для компиляции и исполнения кода на некотором функциональном языке, а также интерфейса к ней. В то время как реализация логики, очевидно, осуществляется более прямолинейно с помощью функционального языка программирования для реализации интерфейса ни один функциональный язык не обладает достаточно развитой экосистемой. Поэтому было решено использовать для реализации платформу .NET, в которую в том числе входят языки C Sharp и F Sharp. Все входящее в платформу .NET объединено общим стандартом для коллекцией, переменных и т.п., что позволяет обращаться к функциям, описанном на функциональном языке (F Sharp в данном случае) из ОО языка C Sharp, на котором было решено запрограммировать программный интерфейс с использованием Windows Presentation Foundation.
Выбор библиотеки NUnit для тестирования обусловлен:
\begin{enumerate}
	\item Интеграции с использованной IDE - JetBrains Rider;
	\item Легкая читаемость методов для проверки поведения программы (напр. \verb|Assert.AreEqual(expected, actual)|)
	\item Возможность аггрегации и отдельного выполнения тестов.
\end{enumerate}
Система контроля версий была выбрана так как Mercurial использовался в модифицируемом проекте с его создания. Выбор среды JetBrains Rider обуславливается удобной навигации по проекту, напр. быстрая навигация к нужным методам и поиск потомков классов с помощью специальных клавиш.

\section{Описание реализованного функционала}

\begin{annotation}
	Приводится описание того, что было реализовано в течении УИР, а также того, что осталось реализовать для завершения начатого проекта. Дается описание реализованного программного обеспечения.
\end{annotation}

В ходе учебной работы была выполнена задача реализации КАМ-машины для перевода кода на некотором функциональном языке в его объектное представление для последующей дебрейнизации данного представления. Также была реализована компиляция дебрейнизированного кода в объектное представление КАМ-инструкций.

Однако, не была реализована обработка и исполнение аппликативно-реляционных конструкций наряду с обычными аппликативными конструкциями, а также осталась невыполненной доработка существующего интерпретатора объектного представления КАМ-инструкций для его совместимости с модифицированным объектным представлением.


\section{Отладка, тестовые примеры, тестирование.}

\begin{annotation}
	В данном подразделе описываются характерные тестовые примеры, для прогона на интеграционных тестах. Приводятся результаты тестирования, включая таблицы и графики. Предоставляется таблица профилирования данной реализации.
\end{annotation}


Для проверки работы функции дебрейнизации объектного представления кода и его компиляции в КАМ-код были использованы следующие тестовые примеры:
\begin{itemize}
	\item $\lambda x.x$. Результат дебрейнизации: \{0\}, код для проверки корректности:
	
	\verb|Assert.That(deAbs.ToString(), Is.EqualTo("{0}"))|
	\item $\lambda xyz.xyz$. Результат дебрейнизации: ((\{2\} \{1\}) \{0\}), код для проверки корректности:
	
	\verb|Assert.That(deAbs.ToString(), Is.EqualTo "(({2} {1}) {0})")|
	\item $\lambda x \lambda y \lambda z.((xz)(yz))$. Результат дебрейнизации: ((\{2\} \{0\}) (\{1\} \{0\})), код для проверки корректности:
	
	\verb|Assert.That(absDe.ToString(), Is.EqualTo "(({2} {0}) ({1} {0}))")|
	
	\item \{0\}. Компиляция с ленивыми вычислениями. Результат компиляции: $Code [Snd; Unfreeze]$.
	
	\item $\lambda xy.x$. Компиляция с ленивыми вычислениями. Результат компиляции: $Code [Fst; Snd; Unfreeze]$, код для проверки корректности дебрейнизации:
	
	\verb|Assert.That(absDe.ToString(), Is.EqualTo "{1}")|
	
	код для проверки корректности компиляции:
	
	\verb|Assert.That(camCode.ToString(), Is.EqualTo "Code [Fst; Snd; Unfreeze]")|
\end{itemize}

%Описываются наиболее характерные тестовые примеры, для прогона на интеграционных тестах. (Да, использование unit-тестирование --- это почти всегда хорошо, основное исключение составляют работы, в которых используемый инструментарий по какой-либо причине в принципе исключает такую возможность. Например, что-нибудь вроде Mathematica.)

%В этом же разделе могут приводится и результаты тестирования, включая таблицы и
%графики. Результаты тестирования могут быть вынесены в отдельный раздел, если
%много текстового материала и/или использована (не совсем) стандартная методика
%тестирования (описание которой также нужно привести).

%\textit{\textbf{Замечание.}} В ПЗ (как УИРа, так и ВКР) следует избегать ситуаций, когда значительную часть основного содержания составляют страницы с иллюстрациями и таблицами, особенно, если такие страницы следуют подряд. В основном тексте следует оставлять лишь самые основные таблицы и рисунки, а остальное --- выносить в приложение.


%В сравнении должно быть отражено, чем полученное ПО выгодно (и невыгодно) отличается от прочих ближайших аналогов. Практика показывает, что аналоги есть всегда. А если нет аналогов, значит есть частичные решения, которые реализуют какие-то части функционала вашей системы. Тут тоже может быть относительно много таблиц и графиков.



\section{Выводы}

\begin{annotation}
	Перечислется полученные результаты. По каждому результату приведены выводы, на сколько он отличается от запланированного программного продукта. Приводятся выводы по результатам тестирования. Указывается производительность реализованного продукта.
\end{annotation}

В качестве результата была получена библиотека, построенная на основе спроектированных в главе 3 интерфейсов, которая переводит объектное представления кода ITerm в Дебрейновскую форму, а затем его компиляция в список КАМ-инструкций. Реализация выделяется наличием полноценной имплементацией концепции среды на основе аппликативной расширенной предструктуры. Полученная реализация была протестирована приведенными в 4.3 тестовыми примерами для проверки корректности написанного кода.
%Следует перечислить, какие практические результаты были получены, а именно: какое программное или иное обеспечение было создано. В число результатов могут входить, например, методики тестирования, тестовые примеры (для проверки корректности/оценки характеристик тех или иных алгоритмов) и др. По каждому результату следует сделать вывод, насколько он отличается от известных промышленных аналогов и исследовательских прототипов.
