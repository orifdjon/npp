\chapter*{Реферат}
\thispagestyle{plain}

Пояснительная записка содержит 31  страниц , 6 рисунков, 3 таблиц.   Количество использованных источников~-- 10.

Ключевые слова: аппликативные выражения, абстрактная машина, КАМ-машина, лямбда-исчисление .

Целью данной работы является реализация абстрактной машины для обработки кода на Я. П., его компиляции и исполнения машиной.

В 1ом разделе даются основные понятия и определения связанные с аппликативными вычислительными системами, а именно $\lambda$-исчисление и комбинаторная логики. Понятие связанной и свободной переменной.
Дается понимание аппликативному подходу программирования. В данном разделе дается определение понятию абстрактная машина. Перечисляются виды абстрактных машин.

В 2ом описывается использование аппликативной вычислительной среды на основе расширенной предструктуры. Приводится описание состава и структуры такой вычислительной среды. Осуществляется разработка модели аппликативно-вычислительной среды на основе расширенной предструктуры. Рассматриваются различные модели для создания визуального отображения. Приводятся ER-диаграммы с указанием доводов для использования их в отображении аппликативных конструкций.Приводится общее описание расширенной аппликативной структуры.Описываются схемы компиляции аппликативных выражений в КАМ для аппликативных выражений в расширенной структуре.

В 3ем разделе приводятся результаты проектирования программных интерфейсов для абстрактной машины и компилятора вычисляемых выражений. Приводятся результаты проектирования интерфейсов и компилатора в виде описания API на UML-диаграммах с подробным описанием функции отдельных частей интерфейсов. Производится проектирование веб-сервиса для компиляции аппликативных выражений в код КАМ и вычислений этого кода. Выдвигаются функциональные требования к данному сервису. Проектируется архитектура веб-приложения.

В 4ом разделе представляется реализация, отлаживание, тестирование настольного приложения КАМ-машины.
Приводятся производительность каждого компонента. Приведены фрагменты программного кода. Настольное приложение написано на платформе .NET. Обосновывается выбор инструментарий для реализации настольного приложения. Веб-сервис реализуется при помощи ASP.NET на сервере и Vue js на клиенте.
