\chapter*{Введение}
\label{sec:afterwords}
\addcontentsline{toc}{chapter}{Введение}

Испания — суверенное государство на юго-западе Европы и частично в Африке, член Европейского союза и НАТО. Испания занимает бо́льшую часть (80 \%) Пиренейского полуострова, а также Канарские и Балеарские острова, имеет общую площадь 504 782 км² (вместе с небольшими суверенными территориями на африканском побережье, городами Сеута и Мелилья), являясь четвёртой по величине страной в Европе (после России, Украины и Франции).

Имеет сухопутные границы с пятью странами:

Португалией на западе Пиренейского полуострова;
Британским владением Гибралтар на юге Пиренейского полуострова;
Марокко в Северной Африке (полуанклавы Сеута, Мелилья и Пеньон-де-Велес-де-ла-Гомера);
Францией на севере;
Андоррой на севере.
Омывается Атлантическим океаном на севере и западе, Средиземным морем на юге и востоке.

