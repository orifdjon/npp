\chapter*{Введение}
\label{sec:afterwords}
\addcontentsline{toc}{chapter}{Введение}

Испания — суверенное государство на юго-западе Европы и частично в Африке, член Европейского союза и НАТО. Испания занимает бо́льшую часть (80 \%) Пиренейского полуострова, а также Канарские и Балеарские острова, имеет общую площадь 504 782 км² (вместе с небольшими суверенными территориями на африканском побережье, городами Сеута и Мелилья), являясь четвёртой по величине страной в Европе (после России, Украины и Франции).

Испания в 2015 году произвела 281 ТВт-ч электроэнергии, 63 ТВт-ч (22\%) из которой было получено от солнечной энергии и ветра, 57 ТВт-час (20\%) из атомной энергии, 53 ТВт-ч (19\%) из угля, 52,5 ТВт-ч (19\%) из природного газ и 31 ТВтч (11\%) из гидроэнергии. Конечное потребление в 2015 году составило 232 ТВтч, около 5000 кВтч на душу населения. Испания импортировала и экспортировала около 15 ТВт-ч электроэнергии в 2015 году. В конце года чистая установленная мощность составляла 107 ГВт, из которых на ядерную энергию пришлось 7,4 ГВт (7\%). Поскольку Испания по существу отделена от энергосистемы ЕС - небольшое количество электроэнергии может быть продано Франции - однако, самообеспечение электроэнергией является важным политическим соображением.

История атомной энергетики в Испании начинается в 1964, когда было начато строительство первой из трех АЭС - Хосе Кабрера, основанный на небольшом водо-водяном реакторе. Позднее, в 1972 году была основана государственная компания ENUSA Industrias Avanzadas SA, задачей которой являлось забота о всей деятельности связанной с атомной энергетикой.

На сегодняшний день на территории Испании функционирует 6 АЭС. Также все топливо для АЭС поставляется из-за границы - в стране отсутствуют какие-либо предприятия занимающиеся добычей или обогащением урана для электростанций.

Данный реферат посвящен описанию истории развития ядерной энергетики в Испании с подробным рассмотрением каждой из функционирующих станций. Также будет проведено рассмотрением общей информации и характеристик используемых на АЭС блоков и реакторов, дана информация о организациях, участвовавших и участвующих в строительстве электростанция. Будут описаны инциденты, произошедшие на АЭС за время их эксплуатации. Рассказано об особенностях эксплуатации и особых требованиях к атомным электростанциям в Испании.