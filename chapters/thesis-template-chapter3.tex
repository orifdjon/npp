\chapter{Проектирование приложения КАМ-машины.}

%В этой главе описывается, что и как было спроектировано. При необходимости, описывается использованная методика проектирования. Сюда же относится описание внешних и внутренних программных интерфейсов, а также форматы и структуры входных и выходных данных.

\begin{annotation}
	В данном разделе приводятся результаты проектирования программных интерфейсов для абстрактной машины и компилятора вычисляемых выражений. Приводятся результаты проектирования интерфейсов и компилатора в виде описания API на UML-диаграммах с подробным описанием функции отдельных частей интерфейсов. Производится проектирование веб-сервиса для компиляции аппликативных выражений в код КАМ и вычислений этого кода. Выдвигаются функциональные требования к данному сервису. Проектируется архитектура веб-приложения.
\end{annotation}

\section{Проектирование программных интерфейсов для абстрактной машины и компилятора вычисляемых выражений}

\begin{annotation}
	В данном подразделе приводятся результаты проектирования программных интерфейсов для абстрактной машины и компилятора вычисляемых выражений. Приводятся результаты проектирования интерфейсов и компилятора в виде описания API на UML-диаграммах с подробным описанием функции отдельных частей интерфейсов.
\end{annotation}

UML-диаграмма интерфейсов, представляющих программную реализацию пары $(D, F)$ - объектов и конструкторов приведена на \ref{pic:umlCAM}.

%\begin{figure} [h!]
%	\begin{center}
%		\includegraphics[width=.8\columnwidth]{./img/umlCAM.png}%
%	\end{center}
%	\caption{Иерархия интерфейсов ITerm}%
%	\label{pic:umlCAM}%
%\end{figure}


\begin{itemize}
	\item ITerm - интерфейс сигнатура, реализуется всеми типами, которые представляют собой лямбда-термы;
	\item IVariable - интерфейс, который должны наследовать все типы, которые являются переменными, имеет поле Name;
	\item IApplicationTerm - интерфейс для реализации аппликации (термообразующая функция), имеет в качестве полей термы участвующих в аппликации, возврат аргументов в качестве кортежа и приведение аппликации к виду кортежа;
	\item IAtomic - интерфейс сигнатура для задания множества объектов;
	\item Data - основное множество объектов, имеет метод equals и является sum type большинства базовых типов (Boolean и т.п.);
	\item ICompositeTerm - является реализацией ITerm, которая используется для конкретной реализации лямбда-термов. Имеет методы получения термообразующей функции, связанных термом переменных и кортежа подтермов;
	\item IFunctionalAbstraction - интерфейс для реализации абстракции, содержит методы для получения тела абстракции и списка связанных ей переменных;
	\item DeLambda - класс реализующий абстракцию и представляющий абстракцию, закодированную по ДеБрейну, содержит поле Body;
	\item DeNumber - класс реализующий число ДеБрейна, появляющееся после кодирования переменной, содержит поле о глубине связывания;
	\item ITff - интерфейс для термообразующих функций, имеет единственный метод создания терма из последовательности поданных на вход ITerm
	\item SimpleLamdaTff - интерфейс для реализации Tff в виде абстракции с одной переменной
	\item MultiLambdaTff - интерфейс для реализации Tff в виде абстракции с несколькими переменной
	\item IApplicationTff - интерфейс для реализации Tff в виде аппликации
\end{itemize}

\section{Выводы}

\begin{annotation}
	В данном разделе были описаны результаты проектирования интерфейсов для создания объектного множеств из расширенной аппликативной предструктуры. Результаты проектирования приведены в виде UML-диаграммы. Также дано описание сигнатур всех методов, участвующих в спроектированных интерфейсах.
\end{annotation}

В данном разделе были описаны результаты проектирования интерфейсов для создания
объектного представления аппликативных выражений. Результаты проектирования
приведены в виде UML-диаграммы. Также дано описание сигнатур всех методов, участвующих в
спроектированных интерфейсах.

%Следует перечислить, какие инженерные результаты были получены, а именно:
%какие программные системы, подсистемы или модули были спроектированы. Следует
%не только назвать полученные архитектуры, но и отметить их отличительные
%особенности.
